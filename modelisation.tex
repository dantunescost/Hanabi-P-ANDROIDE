\documentclass{article}
\usepackage[utf8]{inputenc}

\title{Approches sur la modelisation du jeu Hanabi}
\date{}
\usepackage{natbib}
\usepackage{graphicx}

\begin{document}

\maketitle

\section{Modélisation exhaustive d'une situation}


    Ici le but est d'identifier chaque situation de jeu possible par un monde de Kripke.
La première étape est de modéliser les mondes liés à la main (combinaison de 4 ou 5 cartes sans
ordre) du joueur, c'est à dire toutes les mains possibles en prenant en compte les cartes vues, déjà jouées ou déjà défaussés. L'agent correspondant au joueur hésite donc entre tous ces ces mondes.
Notre strucure de Kripke est donc:
\smallbreak
$   \mu=\{U,\{R_{1}\},I \}  $
$   U=\{M_{1},...M_{m} \}   $

$R$: A ce stade, $R_{1}$ connecte tous les mondes

$I$: On note    $C_{j,c,v}$ la variable affirmant que le joueur j possède une carte de couleur c et de valeur v. Pour n joueurs, un monde $M_{i}$ contiendra donc toujours un nombre de variables $nv=n*4 | n>3$ ou $nv=n*5 | n<4$
\smallbreak
A ce niveau de modélisation, on peut déjà envisager des prises de décisions probabilistes simplement en comptant le nombre de mondes dans lesquels faire telle action aurait un impact positif.

Ensuite, il est également possible pour le joueur de modéliser les mondes entre lesquels il pense qu'un autre joueur hésite. Appelons le joueur qui réfléchit joueur 1 et celui sur lequel il porte sa
réflexion joueur 2. Cela pose problème car ce joueur 2 dispose d'informations que n'a pas le joueur 1 (ses propres cartes), mais on peut tout de même éliminer des mondes grâce à l'information
commune aux 2 joueurs (les cartes des autres joueurs, les cartes jouées, les cartes défaussées, les indices donnés). Ce raisonnement peut être appliqué pour chaque autre joueur.
On a donc maintenant:

$   \mu'=\{U',\{R_{1},...,R_{n}\},I' \} $

$   U'=\{M_{1},...M_{m'} \} $


Pour les relations $R_{i}$, il faut se dire que le joueur i n'hésite qu'entre les mondes où toutes les cartes sont identiques sauf les siennes.
\smallbreak
$   \forall i \in [2,...,n] ,  \forall (M_{j},M_{k}) \in U'*U',$
\smallbreak
$R_{i}(M_{j},M_{k}) 
\equiv
\forall l \in [1,...,n]|l \ne i,  \forall C_{l,c1,v1} \in  I(M_{j}), \forall C_{l,c2,v2} \in  I(M_{k}), c1=v1, c2=v2    $
\newpage
\section{Modélisation carte par carte}

Cette fois, l'objectif est de créer des structures de Kripke pour chaque carte plutôt que pour chaque situation. Ainsi, sans information, il y aurait 25 mondes pour chaque carte. On peut ensuite tenter d'appliquer les mêmes raisonnements que dans la modélisation exhaustive.

Univers de la i-ème carte du joueur j:

$   \mu_{i,j}=\{U,\{R_{1},...,R_{n}\},I \}  $

$   U=\{M_{1},...M_{m} \}   $

$I$: chaque monde contient 2 variables, une pour la couleur et une pour la valeur. 

$\forall M \in U, I(M)=(c,v)|c\in[rouge,bleu,vert,jaune,blanc],v\in[1,...,5]$
\smallbreak
La question qui se pose ensuite est donc qu'est-ce qui différencie ces deux approches ?
Qu'en est-il de la complexité des opérations à faire pour éliminer des mondes dans chacune d'entre elles ? Autrement dit, existe-t-il des situations où l'approche carte par carte a besoin de faire plus
d'opérations pour supprimer des mondes ?
\end{document}
