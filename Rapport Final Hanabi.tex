\documentclass[11pt, letterpaper]{article}
\usepackage[utf8]{inputenc}
\usepackage{graphics}
\usepackage{graphicx}
\usepackage[francais]{babel}
\usepackage[T1]{fontenc} 
\usepackage[top=3cm, bottom=3cm, left=2cm, right=2cm]{geometry}
\usepackage{float}
\usepackage{url}
\usepackage{fancyhdr}
\usepackage[official]{eurosym}

\newcommand{\hmark}{\rule{\linewidth}{0.5mm}}

\begin{document}
\pagestyle{fancy}

\renewcommand{\headrulewidth}{1pt}
\rhead{}

\begin{titlepage}

\centering
\begin{figure}[t]
\begin{center}
\includegraphics[scale = 0.2]{upmc.png}
\end{center}
\end{figure}

\begin{center}
\par\vspace {2.5cm}
{\Huge\textbf{Rapport final : Hanabi\\*[0.25cm]}}   
\par\vspace {2cm}
{\Large\textbf{Université Pierre et Marie Curie\\ Projet ANDROIDE\\
               2015-2016}}                   
\par\vspace {2cm}
{\Large{Antunes Costa Gonçalves Daniel}}\\
{\Large{Assmann Catalina}}\\ 
{\Large{Hubert Cédric}}\\ 
{\Large{Wolfrom Matthieu}}\\
\end{center}

\end{titlepage}

\newpage

\tableofcontents

\newpage
\pagenumbering{arabic}

\section*{Présentation du projet}
\addcontentsline{toc}{section}{Présentation du projet}


\subsection*{Contexte}

\noindent 
Ce projet se déroule dans le contexte de l'UE Projet de première année de Master ANDROIDE de l'UPMC. Nous sommes un groupe de quatre étudiants qui doit mener à terme un projet proposé par leurs encadrants. Ce projet repose sur l'utilisation d'une intelligence artificielle qui exploite la logique épistémique afin d'appliquer la meilleure stratégie possible pour le jeu Hanabi.

\subsection*{Présentation du jeu Hanabi}

\subsection*{Objectifs}

\subsection*{Description de l'existant}

\section {Expression des besoins}

\subsection{Besoins fonctionnels}

\subsubsection{L'interface Graphique}

\subsection{Besoins non fonctionnels}

\subsection{Critères d'acceptabilité du produit}

\section{Déroulement du projet}

\subsection{Planification}

\section{Analyse et conception}

\subsection{Tâches}

\subsection{Architecture}

\subsection{Diagrammes de classe}

\section{Documentation}

\subsection{Cahier des charges}

\subsection{Manuel d'utilisation}

\section{Modélisation logique}

\section{Tests}





\end{document}
